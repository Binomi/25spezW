% Für Bindekorrektur als optionales Argument "BCORfaktormitmaßeinheit", dann
% sieht auch Option "twoside" vernünftig aus
% Näheres zu "scrartcl" bzw. "scrreprt" und "scrbook" siehe KOMA-Skript Doku
\documentclass[12pt,a4paper,titlepage,headinclude,bibtotoc]{scrartcl}


%---- Allgemeine Layout Einstellungen ------------------------------------------

% Für Kopf und Fußzeilen, siehe auch KOMA-Skript Doku
\usepackage[komastyle]{scrpage2}
\pagestyle{scrheadings}
\automark[section]{chapter}
\setheadsepline{0.5pt}[\color{black}]

%keine Einrückung
\parindent0pt

%Einstellungen für Figuren- und Tabellenbeschriftungen
\setkomafont{captionlabel}{\sffamily\bfseries}
\setcapindent{0em}

\usepackage{caption}

%---- Weitere Pakete -----------------------------------------------------------
% Die Pakete sind alle in der TeX Live Distribution enthalten. Wichtige Adressen
% www.ctan.org, www.dante.de

% Sprachunterstützung
\usepackage[ngerman]{babel}

% Benutzung von Umlauten direkt im Text
% entweder "latin1" oder "utf8"
\usepackage[utf8]{inputenc}

% Pakete mit Mathesymbolen und zur Beseitigung von Schwächen der Mathe-Umgebung
\usepackage{latexsym,exscale,amssymb,amsmath}

% Weitere Symbole
\usepackage[nointegrals]{wasysym}
\usepackage{eurosym}

% Anderes Literaturverzeichnisformat
%\usepackage[square,sort&compress]{natbib}

% Für Farbe
\usepackage{color}

% Zur Graphikausgabe
%Beipiel: \includegraphics[width=\textwidth]{grafik.png}
\usepackage{graphicx}

% Text umfließt Graphiken und Tabellen
% Beispiel:
% \begin{wrapfigure}[Zeilenanzahl]{"l" oder "r"}{breite}
%   \centering
%   \includegraphics[width=...]{grafik}
%   \caption{Beschriftung} 
%   \label{fig:grafik}
% \end{wrapfigure}
\usepackage{wrapfig}

% Mehrere Abbildungen nebeneinander
% Beispiel:
% \begin{figure}[htb]
%   \centering
%   \subfigure[Beschriftung 1\label{fig:label1}]
%   {\includegraphics[width=0.49\textwidth]{grafik1}}
%   \hfill
%   \subfigure[Beschriftung 2\label{fig:label2}]
%   {\includegraphics[width=0.49\textwidth]{grafik2}}
%   \caption{Beschriftung allgemein}
%   \label{fig:label-gesamt}
% \end{figure}
\usepackage{subfigure}
\usepackage{adjustbox}

% Caption neben Abbildung
% Beispiel:
% \sidecaptionvpos{figure}{"c" oder "t" oder "b"}
% \begin{SCfigure}[rel. Breite (normalerweise = 1)][hbt]
%   \centering
%   \includegraphics[width=0.5\textwidth]{grafik.png}
%   \caption{Beschreibung}
%   \label{fig:}
% \end{SCfigure}
\usepackage{sidecap}

% Befehl für "Entspricht"-Zeichen
\newcommand{\corresponds}{\ensuremath{\mathrel{\widehat{=}}}}

%Für chemische Formeln (von www.dante.de)
%% Anpassung an LaTeX(2e) von Bernd Raichle
\makeatletter
\DeclareRobustCommand{\chemical}[1]{%
  {\(\m@th
   \edef\resetfontdimens{\noexpand\)%
       \fontdimen16\textfont2=\the\fontdimen16\textfont2
       \fontdimen17\textfont2=\the\fontdimen17\textfont2\relax}%
   \fontdimen16\textfont2=2.7pt \fontdimen17\textfont2=2.7pt
   \mathrm{#1}%
   \resetfontdimens}}
\makeatother

%Si Einheiten
\usepackage{siunitx}

%c++ Code einbinden
\usepackage{listings}
\lstset{numbers=left, numberstyle=\tiny, numbersep=5pt}

%Differential
\newcommand{\dif}{\ensuremath{\mathrm{d}}}

%Boxen,etc.
\usepackage{fancybox}
\usepackage{empheq}

%Fußnoten auf gleiche Seite
\interfootnotelinepenalty=1000

%Dateien aus Unterverzeichnissen
\usepackage{import}

%Bibliography \bibliography{literatur} und \cite{gerthsen}
%\usepackage{cite}
\usepackage{babelbib}
\selectbiblanguage{ngerman}

\begin{document}

\begin{titlepage}
\centering
\textsc{\Large Anfängerpraktikum der Fakultät für
  Physik,\\[1.5ex] Universität Göttingen}

\vspace*{4.2cm}

\rule{\textwidth}{1pt}\\[0.5cm]
{\huge \bfseries
  Die spezifische\\[1.5ex]
  Wärme}\\[0.5cm]
\rule{\textwidth}{1pt}

\vspace*{3.0cm}

\begin{Large}
\begin{tabular}{ll}
Praktikant:
 	&  Felix Kurtz\\
 	&  Michael Lohmann\\

E-Mail: 
	&  felix.kurtz@stud.uni-goettingen.de\\
	& m.lohmann@stud.uni-goettingen.de\\

 Betreuer: & Phillip Bastian\\
 Versuchsdatum: &  13.03.2015\\
\end{tabular}
\end{Large}

\vspace*{0.8cm}

\begin{Large}
\fbox{
  \begin{minipage}[t][2.5cm][t]{6cm} 
    Testat:
  \end{minipage}
}
\end{Large}

\end{titlepage}

\tableofcontents

\newpage

\section{Einleitung}
\label{sec:einleitung}
Die spezifische Wärmespeicherkapazität ist eine wichtige Materialkonstante, da sie für viele alltäglichen Dinge essentiell ist.
Als Beispiel ist hier die Isolation zu nennen, die die Heizkosten moderat halten.
Hierfür ist es wichtig, Stoffe zu finden, die gut für diese Aufgabe geeignet sind.
Ein Versuch um Materialien zu charakterisieren wurde hier durchgeführt.

\section{Theorie}
\label{sec:theorie}

\subsection{Debye-Modell}
\begin{align}
	c_m=9R \left(\frac{T}{\theta_D}\right)^3\cdot\int\limits_0^\frac{\theta_D}{T} \frac{x^4 e^x}{(e^x-1)^2} \,\dif x
	\label{eq:debye}
\end{align}

\subsection{Wärmekapazität}
\begin{align}
	c_M=\frac{M}{m}\cdot\frac{P}{\frac{\dif T}{\dif t}|_\text{erw}+\frac{\dif T}{\dif t}|_\text{abk}}
	\label{eq:molWaerme}
\end{align}

\section{Durchführung}
\label{sec:durchfuehrung}

\section{Auswertung}
\label{sec:auswertung}
\subsection{Temperaturverläufe}
Das Thermoelement gibt eine Spannung in Millivolt zurück.
Diese kann man mit der folgenden Formel und ihrer Fehlerformel in eine Temperatur umrechnen:
\begin{align}
	T[\si{\celsius}]&=0.219+20.456 \cdot U - 0.302\cdot U^2+0.009\cdot U^3 \,, \\
	\sigma_T&=(20.456 - 0.604\cdot U+0.027\cdot U^2)\cdot \sigma_U\,.
\end{align}
Dabei nehmen wir eine Ungenauigkeit von $\sigma_U=0.02\,$mV an.
In Abbildung \ref{fig:Raumtemp} und \ref{fig:Stickstofftemp} ist die Temperatur der beiden Materialien Aluminium und Beryllium gegen die Zeit aufgetragen -- zuerst für Raumtemperatur, dann für Stickstofftemperatur.
Man kann gut erkennen, wann geheizt wurde und wann sich der Körper wieder abkühlt.
Für die Stickstofftemperatur-Messung von Beryllium wurde der Referenzkontakt nicht in Eiswasser getaucht.
Deshalb ist die Referenztemperatur nicht $0\si\celsius$, sondern Raumtemperatur $T(0.88)=18\si\celsius$.
\begin{figure}[!htb]
	\centering
	\input{Raumtemp.tex}
	\caption{Raumtemperatur: Erhitzen und Abkühlen von Aluminium und Beryllium}
	\label{fig:Raumtemp}
\end{figure}

\begin{figure}[!htb]
	\centering
	\input{Stickstofftemp.tex}
	\caption{Stickstofftemperatur: Erhitzen und Abkühlen von Aluminium und Beryllium}
	\label{fig:Stickstofftemp}
\end{figure}

\subsection{Widerstand}
Den Widerstand des Kupferdrahtes kann man leicht berechnen, indem man die Heizspannung $U$ durch die Stromstärke $I$, die konstant bei$0.5\,$A liegt, teilt:
\begin{align}
	R&=\frac{U}{I}\,,\\
	\sigma_R&=\frac{\sigma_U}{I}\,.
\end{align}
Wir nehmen einen Fehler der Heizspannung von $\sigma_U=0.4\,$V an, da sich die Spannung während 30 Sekunden stark ändern konnte.
Wenn die Spannung mit dem Schalter verdreifacht wurde, verdreifacht sich der Fehler auch auf $\sigma_U=1.2\,$V.
In Abb.\ref{fig:Widerstand} ist der Widerstand des Drahtes während der vier Heizvorgänge gegen die Temperatur aufgetragen.
\begin{figure}[!htb]
	\centering
	\input{Widerstand.tex}
	\caption{Widerstand des Cu-Drahtes}
	\label{fig:Widerstand}
\end{figure}

\subsection{Leistung}
Die Heizleistung kann man mit der Folgenden Formel und ihrer Fehlerformel berechnet werden:
\begin{align}
	P&=UI\,,\\
	\sigma_P&=I\cdot\sigma_U\,.
\end{align}
Es wird der gleiche Fehler $\sigma_U$ wie zuvor auch verwendet.
Nun kann die elektrische Leistung während des Heizvorgangs gegen die Temperatur aufgetragen werden (vgl. Abb.\ref{fig:Leistung}).
\begin{figure}[!htb]
	\centering
	\input{Leistung.tex}
	\caption{Beim Heizen hineingesteckte Leistung}
	\label{fig:Leistung}
\end{figure}

\subsection{molare Wärmekapazität}
Für das Erwärmen wird ein linearer Zusammenhang zwischen Temperatur und Zeit erwartet: $T(t)=a\cdot t +b$.
Dann ist $\frac{\dif T}{\dif t}|_\text{erw}=a$.
Für das Abkühlen kann man einen exponentiellen Abfall der Temperatur mit der Zeit annehmen.
Zudem wird sich die Temperatur dem thermischen Gleichgewicht angleichen: $T(t)=T_0+(T_1-T_0)\cdot e^{-\lambda t}$.
Somit ist $\frac{\dif T}{\dif t}|_\text{abk}=-\lambda\cdot (T-T_0)$.
Dabei ist $T_0$ die Temperatur des thermischen Gleichgewichtes.
Setzt man dies für die molare Wärmekapazität nach \eqref{eq:molWaerme} ein, ergibt sich
\begin{align}
	c_m&=\frac{M}{m}\frac{P}{a+\lambda (T-T_0)}
	\label{eq:c_m}\\
	\sigma{c_m}&=\frac{M}{m}\sqrt{\frac{\sigma_P^2}{(a+\lambda (T-T_0))^2}+\frac{P^2}{(a+\lambda (T-T_0))^4}\cdot (\sigma_a^2+(T-T_0)^2 \sigma_\lambda^2+\lambda^2\sigma_T^2)}\,.
	\label{eq:sigma_c_m}
\end{align}
Die molare Masse von Aluminium beträgt $M=26.982\,\si{\gram\per\mol}$, von Beryllium $M=9.012\,\si{\gram\per\mol}$.
Die Masse des Aluminium-Körpers ist $m=52.5\,\si\gram$, die des Beryllium-Körpers ist $m=43.0\,\si\gram$.\\
Um die Temperatur $T_0$ zu ermitteln, werden die Werte vor dem Erhitzen verwendet.
Diese waren bei Zimmertemperatur erwartungsgemäß konstant.
Somit liegt die Zimmertemperatur bei $T_0(0.88\,\text{mV})=18\si{\celsius}$.
Bei dem zweiten Versuchsteil wurde vor dem Erhitzen noch nicht das thermische Gleichgewicht erreicht.
Man nimmt wieder an, dass sich die Temperatur exponentiell abfallend an die Gleichgewichtstemperatur nähert.
Mit einem $\chi^2$-Fit ergibt sich für die Versuchsreihe mit Aluminium $T_0=110\si\celsius$, für die mit Beryllium $T_0=106\si\celsius$.
So kann man jetzt $\lambda$ bestimmen, indem man die Temperaturdifferenz zu $T_0$ während des Abkühlvorgangs logarithmisch gegen die Zeit aufträgt.
Es ergibt sich eine Gerade, deren Steigung mit einer linearen Regression bestimmt werden kann.
Die Werte für $a$ und $\lambda$ der 4 Versuchsreihen befinden sich in Tabelle \ref{tab:fitwerte}.
\begin{table}[!htb]
	\centering
	\begin{tabular}{|c|c|c|}
		\hline
		& $a~[10^{-3}\cdot\si{\kelvin\per\second}}]$ & $\lambda~[10^{-5}\cdot\si{\per\second}}]$\\
		\hline
		Al RT	& $	303	\pm	2	$ & $	87.8	\pm	0.6	$ \\
		Be RT	& $	188.5	\pm	1.2	$ & $	51.4	\pm	0.6	$ \\
		Al Stickstoff	& $	74.85	\pm	0.23	$ & $	158.8 \pm 0.4	$ \\
		Be Stickstoff	& $	109	\pm	4	$ & $	171 \pm 5$ \\
		\hline
	\end{tabular}
	\caption{Temperaturverläufe: gefittete Parameter}
	\label{tab:fitwerte}
\end{table}
Mit diesen und Formel \eqref{eq:c_m} sowie \eqref{eq:sigma_c_m} kann man nun die molare Wärme berechnen und gegen die Temperatur auftragen (vgl. Abb.\ref{fig:molWaerme}).
Nach Dulong-Petit sollte sich der konstante Wert $3R$ ergeben, nach dem Debye-Modell erwartet man einen Verlauf nach \eqref{eq:debye}.
Dabei liegt die Debye-Temperatur nach \cite[S.226]{prakti} für Aluminium bei $\theta_D=428\,\si\kelvin$, für Beryllium bei $\theta_D=1440\,\si\kelvin$.

\begin{figure}[!htb]
	\centering
	\input{waerme.tex}
	\caption{molare Wärmekapazität bei verschiedenen Temperaturen für Aluminium und Beryllium sowie Vergleich mit dem Dulong-Petit-Wert und den Verläufen nach Debye}
	\label{fig:molWaerme}
\end{figure}

\section{Diskussion}
\label{sec:diskussion}
\subsection{Temperaturverläufe}
Beim Heizen bei Zimmertemperatur (Abb.\ref{fig:Raumtemp})fällt auf, dass die beiden Körper relativ schnell die Maximaltemperatur von $80\si\celsius$ erreichen.
Damit man mehr Messwerte erhält, hätte man in kleineren Zeitintervallen messen müssen oder bei einer kleineren Stromstärke heizen sollen.
Die wenigen Messwerte beeinflussen die Bestimmung der molaren Wärmekapazität eventuell negativ.

An den Temperaturverläufen, die bei Stickstoffkühlung aufgenommen wurden, also Abb.\ref{fig:Stickstofftemp}, kann man erkennen, dass vor Anfang der Messung noch nicht die Siedetemperatur von Stickstoff $-196\si\celsius\corresponds 77.15\,\si\kelvin$ erreicht wurde.
Das thermische Gleichgewicht ist auch nicht erreicht worden.
Um bessere Messwerte zu erhalten, hätte man also noch länger warten müssen oder mehr Stickstoff verwenden müssen.
Der Knick während des Erwärmens von Beryllium in Abb.\ref{fig:Stickstofftemp} lässt sich damit erklären, dass die Heizspannung danach mit dem Schalter verdreifacht wurde und so erstmal kleiner gedreht werden musste.

Bei Beryllium hätte man länger den Abkühlvorgang beobachten müssen, um den erwarteten exponentiellen Abfall an die Gleichgewichtstemperatur zu beobachten.
Die Messwerte sehen nämlich ziemlich linear aus.

\subsection{Widerstand und Leistung}
Bei den Widerstandswerten aus Abb.\ref{fig:Widerstand} erkennt man, dass sie bei dem Aufbau für Aluminium stets geringer sind als bei dem für Beryllium.
Dies lässt sich damit erklären, dass der verwendete Kupferdraht bei Beryllium länger oder dünner ist.
Dies erhöht den Widerstand.
Außerdem ist der Widerstand stark temperaturabhängig.
Interpoliert man den Widerstand bei noch niedrigeren Temperaturen, müsste man einen Punkt erreichen, an dem der Draht keinen Widerstand hat.
Dies zu untersuchen wäre auch interessant gewesen, man hätte dazu aber nicht mehr Stickstoff als Kühlung nehmen können.


\subsection{molare Wärmekapazität}

\bibliography{literatur}
\bibliographystyle{babalpha}

\end{document}
